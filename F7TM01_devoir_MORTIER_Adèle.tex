\documentclass[a4paper,10pt]{article}
\usepackage[utf8]{inputenc}
\usepackage[french]{babel}
\usepackage{frbib}
\usepackage{french}
\usepackage{graphicx}
\usepackage{geometry}
\geometry{
	a4paper,
	left=20mm,
	top=20mm,
}


\title{F7TM01 -- Théories et poétiques du récit\\
	 D. Sallenave, \textit{Le Don des morts. Sur la littérature}\\ \vspace{0.3cm}
	 \small Commentaire de texte}
\author{Adèle Mortier}

\begin{document}
	\maketitle
	\nocite{*}
	\section*{Introduction}
	\section{Fonctions du roman}
		\subsection{Fonction de description}
			Description : individuelle ou totzalisante? Bien ou mal?
			Sur la transposition et Aristote : Paolo Tortonese, l'homme en actions
		
		
			Aristote p 51, 27 (cf images) : la représentation est positive, source de vertu pour l'homme, "[…] le rôle du poète est de dire non pas ce qui a réellement eu lieu mais ce à quoi on peut s’attendre, [...]  la poésie dit plutôt le général, l’histoire le particulier" (la poétique chap IX) Recherhce d'une image harmonieuse de la vie.
			
			
			
			Ce qu’Auerbach appelle style « créaturel » (kreatürlich), c’est le style qui se rapporte à l’être humain comme créature finie, sujette à la souffrance et à la mort. Ce style se développe chez Saint-François d’Assise et chez les ordres mendiants. Rabelais en y compris pour la
			truculence et la bouffonnerie. Il exprime la complexité contradictoire de la nature humaine,
			mais aussi le triomphe vitaliste et la pulsation dynamique de la réalité corporelle
			
			
			
			
			
			Platon (qui dénonce l'épopée)
			
			selon \cite{Platon}:
			\begin{center}
				\footnotesize
				\begin{minipage}{0.7\textwidth}
					``Or donc, poserons-nous en principe que tous les poètes, à commencer par Homère, sont de simples imitateurs
					des apparences de la vertu et des autres sujets qu’ils traitent, mais que, pour la vérité, ils n’y atteignent pas''
				\end{minipage}
			\end{center}
			Le muthos par opposition au logos n'est pas à même de rendre la vérité, et donc a fortiori, la vérité de l'existence des hommes. (+Hegel éventuellement? L'esthétique)
			
			
			
			\cite{Blanchot1959}
			\begin{center}
				\footnotesize
				\begin{minipage}{0.7\textwidth}
					``Avec le roman, ce qui est au premier plan, c'est la navigation préalable [...] cette navigation est une histoire toute humaine, elle intéresse le temps de hommes, est liée aux passions des hommes, elle a lieu réellement et elle est assez riche et assez variée pour absorber toutes les forces et toute l'attention des narrateurs''
				\end{minipage}
			\end{center}
			\cite{Rabate2010} :
			\begin{center}
				\footnotesize
				\begin{minipage}{0.7\textwidth}
					``Cette capacité à figurer la vie d’un individu comme une particularité compréhensible
					et offerte à l’appréciation d’autrui me semble être ce que la littérature s’est justement proposée
					comme but et ambition à partir du XIXe siècle [...]''
				\end{minipage}
			\end{center}
			\cite{Lukacs191649}
			\begin{center}
				\footnotesize
				\begin{minipage}{0.7\textwidth}
					``Le roman est l'épopée d'un temps où la totalité extensive de la vie n'est plus donnée de manière immédiate, d'un temps pour lequel l'immanence du sens à la vie est devenue problème mais qui, néanmoins, n'a pas cessé de viser à la totalité'' p.49
				\end{minipage}
			\end{center}
			
			\cite{Lukacs191654}
			\begin{center}
				\footnotesize
				\begin{minipage}{0.7\textwidth}
					``L'épopée façonne une totalité de vie achevée par elle-même, le roman cherche à découvrir et à édifier la totalité secrète de la vie [...] Ainsi l'esprit fondamental du roman, celui qui en détermine la forme, s'objective comme psychologie des héros romanesques : ces héros sont toujours en quête [...]'' p.54
				\end{minipage}
			\end{center}
		\subsection{Fonction d'explication}
			citer Husserl, Sartre
			\cite{Rabate2010} :
			\begin{center}
				\footnotesize
				\begin{minipage}{0.7\textwidth}
					``Affirmer l’indéfini contre le défini, une vie contre la vie, c’est encore déjouer la quête d’un sens
					unique, contourner dès le départ l’écueil redoutable [...] de voir dans la littérature l’affirmation univoque d’une morale. C’est donc plutôt de l’impossibilité d’une réponse globale qu’il faudra partir ''
				\end{minipage}
			\end{center}
			Selon Rabate, aucune tentative d'explication ne pourra aboutir, ni "explication", ni "justification".
	\section{Prérogatives du roman}
		\subsection{Le roman comme partie de la vie}
			selon \cite{Pingaud1992}
			\begin{center}
				\footnotesize
				\begin{minipage}{0.7\textwidth}
					``Dès l'instant où je raconte, ce que je raconte, arraché au temps vécu, ne fait plus qu'un avec le temps du récit [...] à partir du moment où on raconte, il n'y a plus, d'une certaine manière, de problème de contenu. C'est le contenu qui est lui-même devenu forme [...] comme tout ce que dit le narrateur, [les lacunes, les amplifications] font partie intégrante d'un temps qui n'est plus le temps abstrait du philosophe, ni non plus le temps concret de l'existence quotidienne, mais leur miraculeuse identification''
				\end{minipage}
			\end{center}
			selon \cite{Blanchot1959}:
			\begin{center}
				\footnotesize
				\begin{minipage}{0.7\textwidth}
					``Le récit n'est pas la relation de l'événement, mais cet événement même''
				\end{minipage}
			\end{center}
			blanchot : intrication du temps vécu et du temps du récit (Proust) : le roman est monde de la vie.
			Blanchot voit le roman comme un temps à part, Sallenave comme un monde a part.
			
			
			
			\cite{Lukacs191654}
			\begin{center}
				\footnotesize
				\begin{minipage}{0.7\textwidth}
					``Un tel péril ne peut être surmonté que si l'on pose comme ultime réalité, en pleine conscience et de façon parfaitement adéquate, ce que le monde a de fragile et d'inachevé, ce qui renvoie en lui à autre chose qui le dépasse'' p.54
				\end{minipage}
			\end{center}
		\subsection{Le roman comme extension de la vie}
			\cite{Barthes1966} :
			\begin{center}
				\footnotesize
				\begin{minipage}{0.7\textwidth}
					``Innombrables sont les récits du monde.C'est d'abord une variété prodigieuse de genres, eux-mêmes distribués entre des substances différentes, comme si toute matière était bonne à l'homme pour lui confier ses récits : le récit peut être supporté par le langage articulé, oral ou écrit, par l'image, fixe ou mobile, par le geste et par le mélange ordonné de toutes ses substances; il est présent dans le mythe, la légende, la fable, le conte, la nouvelle, l'épopée, l'histoire, la tragédie, le drame, la comédie, la pantomime, le tableau peint [...], le vitrail, le cinéma, les comics, le fait divers, la conversation.''
				\end{minipage}
			\end{center}
			On dépasse ce que Panofsky appelat  ‘imitations par copie" [mimésis eikastiké], qui reproduisent les contenus de la réalité qui se donne à la perception sensible,
			ce qui revient à redoubler inutilement le monde matériel, qui n’est lui-même qu’une imitation des
			idées \cite{Panofsky1983}
			
			
			selon \cite{Blanchot1959}
			\begin{center}
				\footnotesize
				\begin{minipage}{0.7\textwidth}
					``ce pouvoir qui fait coïncider, en un même point fabuleux, le présent, le passé et même [...] l'avenir '' p.25
				\end{minipage}
			\end{center}
			
			Ricoeur, ipséité
			Jauss?
	\section{Discussion}
	\section*{Conclusion}








truc sur husserl (préface de depraz)
\begin{center}
	\footnotesize
	\begin{minipage}{0.7\textwidth}
		``Dès lors, le monde de la vie procède d’une attitude pré-théorique au sens de non idéalisante, et il se rapproche selon ce critère de l’attitude naturelle elle-même. [...] Le « monde de la vie » n’est donc pas le monde objectif de la science, il est bien plutôt le fondement oublié du sens de
		la science elle-même.''
	\end{minipage}
\end{center}

\cite{Blanchot1959}, accent mis sur la narration, le récit comme aventure humaine réelle. Aspect expérimental mis en avant également : le roman est une aventure, une Odyssée. Dimension de jeu absente de notre texte : aspect léger. Scepticisme fa ce au pouvoir de vérité du récit




Lukacs : d'accord sur le fait que le roman emprunte à l'épopée, mais que le contexte a changé.
Approche normative et exiologique de la tragédie et de l'épopée. Le roman est la continuation de l'épopée et hérite d'elle la grandeur que le vers lyrique n'est plus à meme de rendre

role du roman par rapport à l'épopée :

(Le monde de la convention n'est d'aucun secours, n'offre aucun réponse sur le sens de la vie t de actions. Or dans la littérature ne subsiste que ce qui est extra-conventionnel)
Exemple deDante qui opère la jonction entre epopee etroman, entre dépersonnalisation t individuation

Exemple de la biographie
\medskip

\bibliographystyle{frplain}
\bibliography{bibliography}

\end{document}
