\documentclass[a4paper,10pt]{article}
\usepackage[utf8]{inputenc}
\usepackage[polutonikogreek, french]{babel}
\usepackage{frbib}
\usepackage{french}
\usepackage{graphicx}
\usepackage{geometry}
\geometry{
	a4paper,
	left=20mm,
	top=20mm,
}
\usepackage{footbib}
\usepackage{textcomp}
\usepackage{graphicx}
\usepackage{hyperref}




\title{F7TM01 -- Théories et poétiques du récit\\
	 D. Sallenave, \textit{Le Don des morts. Sur la littérature}\\ \vspace{0.3cm}
	 \small Commentaire de texte}
\author{Adèle Mortier}

\begin{document}
	\maketitle
	\nocite{*}
	\tableofcontents
	
	\section*{Introduction}
		Le roman est sans doute la forme de littérature qui nous est la plus intime, nous accompagnant en voyage, au coucher, ou durant nos trajets quotidiens. Si le roman nous parle plus que le théâtre ou la poésie, c'est peut-être et avant tout parce que le roman parle de nous; en tout cas c'est la thèse que soutient Danièle Sallenave dans son essai intitulé \textit{Le Don des morts. Sur la littérature}\footcite{Sallenave1991}, paru en 1991.\\
		Danièle Sallenave est tout à la fois une écrivaine de formation classique et membre de l'Académie française, une penseuse du communautarisme (religieux) et une chroniqueuse sur France Culture. Sa connaissance de la littérature antique l'a conduite à explorer les origines grecques du roman\footnote{alors même que le nom de ``roman'' pour désigner une composition littéraire écrite en langue romane est apparu autour du XIIème siècle...}, et à tisser des liens (formels, fonctionnels) entre roman, tragédie et épopée. Elle entend ainsi définir l'essence propre du roman, qui, selon elle ``n'est pas un genre littéraire", mais plutôt, comme nous allons le voir, un espace tout à la fois de réflexion de la vie, de réflexion sur la vie, et d'extension de la vie.\\
		Jouant sur la dialectique de l'existentialisme et du formalisme, de la description et de l'explication, de l'appartenance et de l'extension, Danièle Sallenave nous invite à nous poser les questions suivantes : en quoi une généalogie du roman permet-elle de le définir comme une forme transcendante ? Quelles sont les fonctions du roman, d'un point de vue philosophique ? Comment le roman se positionne-t-il par rapport à la vie ``réelle'' ? Notre plan suivra, détaillera et discutera ce triple questionnement -- qui épouse par ailleurs la progression du texte.
		
	\section{Le roman comme ``forme'' littéraire}
		Dans la première partie de notre extrait, qui couvre la majorité du premier paragraphe (l. 1-11), D. Salenave adopte un point de vue généalogique et plutôt formaliste, le but étant de soutenir la thèse suivante : ``il y a une essence trans-historique du roman'' (l.3-4).
			\subsection{Des origines discutables}
				Dès les premières lignes de l'extrait, D. Sallenave met en avant l'incertitude qui pèse sur les origines du roman : ``Qu'on veuille en dater l'émergence avec celle de l'homme moderne ou qu'on en fasse reculer l'apparition en des temps plus reculés avec le roman grec et latin, dont après tout la définition et les formes ne sont pas si éloignées des nôtres [...]'' (l.7-9).On note en particulier le grand écart historique qui sépare ces deux naissances hypothétiques : le roman dans son acception ``moderne'' (roman baroque ou roman picaresque) daterait du XVIème siècle ou du XVIIème siècle, tandis que le roman dans son acception ``antique'' remonterait environ au Ier siècle après J.-C. La première hypothèse, celle du roman ``moderne'' s'appuierait davantage sur une parenté de contenu, tandis que la seconde, celle du roman ``antique'' ``dont après tout la définition et les formes ne sont pas si éloignées des nôtres [...]'' (l.9), aurait une justification de nature plus formelle.  En effet, le roman grec serait, dans le monde antique, la première forme de littérature en prose développant par ailleurs une intrigue cohérente et réaliste \footcite{WikiRomanGrec}.\\
				Néanmoins, D. Sallenave semble nous dire que cette incertitude temporelle importe bien peu, dans la mesure ou l'essence du roman est restée la même à travers les âges (en particulier, on pense à la veine baroque du roman moderne qui a conservé les traits du roman grec...). C'est cette conservation à travers les époques et malgré des changements de formes, des hybridations, qui semble donner au roman une dimension plus vaste que celle propre aux ``genres littéraires'' classiques comme la tragédie, évoquée l.4-5. Cette preuve de la portée ``trans-historique'' et par là même transcendante du roman prouve que ce concept dépasse le cadre des œuvres : c'est, dans l'approche formaliste, une ``forme littéraire'' plus qu'un genre littéraire.
			\subsection{Une généalogie paradoxale}\label{genealogie}
				Si le roman n'est pas un ``genre", l'auteure ne manque cependant pas d'évoquer les genres qui gravitent autour de lui, qui l'ont inspiré et/ou qui partagent avec lui certaines caractéristiques. Le roman est ainsi qualifié de ``successeur de l'épopée'' (l.10), et partage sa ``mission'' (l.10), une mission qui est aussi celle de la tragédie. Pour mieux situer de quoi il est question ici, on peut se référer à la lecture qu'ont fait Jean Lallot et Roselyne Dupont-Roc de \textit{La Poétique} d'Aristote\footcite{Lallot1980} :
				\begin{figure}[h]
					\begin{center}
						\includegraphics[width=150px]{aristote_genres.png}
						\caption{Parentés entre les trois principaux genres du théâtre antique (\textit{La Poétique}, édition du Seuil, chap. V, note 1.)}
					\end{center}
				\end{figure}\\
				Comme on peut le voir, tragédie et épopée sont en quelque sorte demi-sœurs, partageant un même objet (un sujet élevé : dilemmes, passions des hommes nobles...), mais possédant des modes différents (mode dramatique pour la tragédie, narratif pour l'épopée). Lallot et Dupont-Roc ont également remarqué que la tragédie occupe dans ce diagramme une position polaire, partageant avec la comédie et l'épopée un caractère alors que comédie et épopée demeurent incommensurables. L'objet dont parlent Lallot et Dupont-Roc se rapproche de la ``mission'' évoquée par Sallenave, quoique le roman ne s'applique pas forcément à des sujets élevés. En outre, le roman partage avec l'épopée le mode narratif. Il serait donc un candidat possible pour rééquilibrer le diagramme et remplir une case vide de la typologie...\\
				Notons également qu'un telle approche mettant en regard mode et objet, forme et fonction, peut être mise en perspective avec celle du formalisme russe vers la fin des années 20. Pour Tynianov, ``l'étude des genres isolés est impossible hors du système dans lequel et avec lequel ils sont en corrélation"; le roman contemporain hériterait ainsi du roman grec de par sa forme prosaïque et de la poésie épique de par sa fonction. Or c'est à peu de choses près ce que nous dit D. Sallenave dans cet extrait.\\
				Néanmoins, D. Sallenave reste sur un registre positiviste lorsqu'elle évoque la tragédie, qui dans notre généalogie s'avère être la tante du roman : ``Il n'en est pas du roman comme de la tragédie classique qui connut son temps d'éclat et finit par sombrer au milieu du XVIIIème siècle [...]'' (l.4-5). Or, dans l'optique de Chklovski et de sa théorie de la ``marche du cavalier"\footcite{Chklovski1973}, il se peut que la déliquescence de la tragédie ait précisément été causée par l'essor de la ``branche cadette", celle de l'épopée et du roman... ``L'art détruit son passé ou demeure vivant, tantôt s'embrasant, tantôt tombant dans la pitoyable existence d'une collection où tout a même valeur [...]''
	
	\section{Le roman et ses fonctions}
		Suite à cette étude du roman comme ``essence'', voyons comment celle-ci se réalise et s'actualise pour remplir différents buts; bref, selon quelles modalités le roman \textit{existe}. Cette explication couvre la fin du premier paragraphe et le deuxième paragraphe (l.10-20), et détaille la visée ontologique (décrire l'être) et étiologique (expliquer le \textit{pourquoi} de l'être) du roman.
		\subsection{Un vecteur de la \textit{mimesis}}
			\subsubsection{Un reflet de la vie}
				La mission la plus élémentaire du roman qui est détaillée dans notre extrait est de refléter la vie humaine : ``le roman ne cesse d'avoir pour tâche et pour sujet nous mêmes, notre existence dans le monde'' (l.13-14). ``Nous mêmes'' nous dit D. Sallenave, s'inspirant d'Aristote, c'est tout à la fois nos ``actions'' et nos ``passions'', ce que nous sommes en actes, et en puissance, notre intériorité et notre extériorité. Aristote expliquait en effet dans sa \textit{Poétique}\footcite{Aristote} : 
				\begin{center}
					\footnotesize
					\begin{minipage}{0.7\textwidth}
						``Une conséquence évidente de tout ceci, c'est que le poète doit être bien plus poète, ou compositeur, de fables que de vers, d'autant plus qu'il n'est réellement poète que parce qu'il imite des actions''\footcite[p.~51]{Aristote}
					\end{minipage}
				\end{center}
				Ici bien sûr Aristote ne parle pas du roman mais de la tragédie et de l'épopée. Cela dit, D. Sallenave comme on l'a vu considère le roman comme fils de l'épopée, celui-ci hérite donc directement de la fonction mimétique. 
				\begin{center}
					\footnotesize
					\begin{minipage}{0.7\textwidth}
						``L'épopée tient à la tragédie en ce qu'elle est comme elle, sauf le mètre, une imitation de actions nobles à l'aide du discours. Mais une différence, c'est que dans l'épopée le mètre est toujours le même, et qu'elle est toujours un récit.''\footcite[p.~27]{Aristote}
					\end{minipage}
				\end{center}
				Dans cet autre extrait est réaffirmée la parenté des ``missions'' entre l'épopée et la tragédie, malgré des différences de mode (cf \ref{genealogie}). D'un point de vue moral, Aristote juge la représentation positive, source de vertu pour l'homme.\\
				Plus proche de nous, Maurice Blanchot évoque aussi cet aspect mimétique du roman, usant d'une métaphore nautique :
				\begin{center}
					\footnotesize
					\begin{minipage}{0.7\textwidth}
						``Avec le roman, ce qui est au premier plan, c'est la navigation préalable [...] cette navigation est une histoire toute humaine, elle intéresse le temps de hommes, est liée aux passions des hommes, elle a lieu réellement et elle est assez riche et assez variée pour absorber toutes les forces et toute l'attention des narrateurs''\footcite{Blanchot1959}
					\end{minipage}
				\end{center}
				On voit ici que l'accent est mis sur les passions humaines, mais aussi sur le ``réel''.\\
				Or, ce que D. Sallenave omet de mentionner dans notre extrait est que la représentation selon Aristote ne se doit pas d'être absolument fidèle au réel :
				\begin{center}
					\footnotesize
					\begin{minipage}{0.7\textwidth}
						``[...] le rôle du poète est de dire non pas ce qui a réellement eu lieu mais ce à quoi on peut s’attendre, [...]  la poésie dit plutôt le général, l’histoire le particulier''\footcite[chap.~IX]{Aristote}
					\end{minipage}
				\end{center}
				Il s'agit avant tout de créer une image de la vie qui soit certes cohérente, mais surtout harmonieuse. Cette notion de l'image ``harmonieuse'' mais non réelle, vivant par le \textit{muthos}, est néanmoins rejetée par Platon -- qui ne se fie qu'au \textit{logos} -- en particulier lorsqu'elle est présente dans l'épopée :
				\begin{center}
					\footnotesize
					\begin{minipage}{0.7\textwidth}
						``Or donc, poserons-nous en principe que tous les poètes, à commencer par Homère, sont de simples imitateurs
						des apparences de la vertu et des autres sujets qu’ils traitent, mais que, pour la vérité, ils n’y atteignent pas''\footcite[Livre~X]{Platon}
					\end{minipage}
				\end{center}
				Cette double interprétation de la \textit{mimesis} est remarquablement expliquée par Myriam Revault d'Allonnes\footcite{Revault1992} :
				\begin{center}
					\footnotesize
					\begin{minipage}{0.7\textwidth}
						``il [Aristote] contracte ou rétrécit la mimesis platonicienne en la faisant porter sur l'humain, et plus précisément encore, sur les hommes agissants et, faudrait-il ajouter, souffrants. En ce sens, on passe d'une \textit{mimesis} généralisée à une \textit{mimesis} restreinte, d'une \textit{mimesis} qui prend la mesure du réel en se référant (selon le critère de la ressemblance) à des formes stables et intelligibles, à une \textit{mimesis} qui se déploie dans le champ de l'agir et du pâtir humain. Chez Platon, la \textit{mimesis} est un mode de la participation, chez Aristote,
						elle explore le possible et donne à voir ce qui pourrait être autre qu'il n'est. ''
					\end{minipage}
				\end{center}
				Nous voyons donc que D. Sallenave ne soutient qu'un point de vue sur la \textit{mimesis}, qui fut largement discuté dès la période antique.
				%(+Hegel éventuellement? L'esthétique)
				%aristote : représentation des hommes ``agissants et souffrants"
				%Sur la transposition et Aristote : Paolo Tortonese, l'homme en actions
				%Ce qu’Auerbach appelle style « créaturel » (kreatürlich), c’est le style qui se rapporte à l’être humain comme créature finie, sujette à la souffrance et à la mort. Ce style se développe chez Saint-François d’Assise et chez les ordres mendiants. Rabelais en y compris pour la truculence et la bouffonnerie. Il exprime la complexité contradictoire de la nature humaine, mais aussi le triomphe vitaliste et la pulsation dynamique de la réalité corporelle
				
			\subsubsection{\textit{Quid} du contexte ?}
				Dans cet extrait, D. Sallenave nous parle du roman et des hommes. Mais l'interprétation qui est faite de la \textit{mimesis} demeure très immanente : les hommes seraient caractérisés de façon satisfaisante par leurs principes internes (leur  passions) ou les manifestations directes des ces principes (leurs actions). La réalité à dépeindre n'est-elle pas plus complexe ? Les hommes ne sont-ils pas aussi influencés par des facteurs exogènes ?\\
				C'est en tout cas la thèse de l'un des grands théoriciens allemands du XXème siècle, Erich Auerbach. Auerbach\footcite{Auerbach1946} prend notamment appui sur l'œuvre de Stendhal pour mettre en évidence les facteurs sociaux et historiques qui pèsent sur la théorie du reflet :
				\begin{center}
					\footnotesize
					\begin{minipage}{0.7\textwidth}
						``Dans la mesure où le réalisme sérieux des temps modernes ne peut représenter l’homme autrement qu’engagé dans une réalité globale
						politique, économique et sociale en constante évolution, Stendhal est son fondateur ''\footcite[p.~459]{Auerbach1946}
					\end{minipage}
				\end{center}
				Dans cette optique, les actions et les passions des hommes n'ont de sens que dans un contexte socio-historique défini, un contexte dans lequel l'homme est ``engagé''.
				\begin{center}
					\footnotesize
					\begin{minipage}{0.7\textwidth}
						``Le traitement sérieux de la réalité contemporaine, l’ascension de vastes groupes humains socialement inférieurs au statut de sujets d’une représentation problématique et
						existentielle, d’une part, - l’intégration des individus et des événements les plus communs
						dans le cours général de l’histoire contemporaine, l’instabilité de l’arrière-plan historique,
						d’autre part, - voilà, croyons-nous, les fondements du réalisme moderne'' \footcite[p.~487]{Auerbach1946}
					\end{minipage}
				\end{center}
				Dans cet autre passage, Auerbach semble adopter la thèse de Sallenave selon laquelle le roman véhicule un représentation ``existentielle'' des hommes, mais il nous dit aussi que cette thèse est incomplète : le roman -- ou à tout le moins le roman ``moderne'' -- doit aussi donner une idée du contexte dans lequel évoluent les personnages.

		\subsection{Visée explicative}
			Mais la ``mission'' du roman selon D. Sallenave ne s'arrête pas à la simple réflexion de la vie humaine : c'est aussi une réflexion \textit{sur} la vie humaine, en d'autres termes, un moyen d'introspection, un moyen de comprendre le pourquoi de nos états mentaux, de nos actions, de notre \textit{existence} en ce monde.\\
			Le roman est ainsi défini comme  ``le lieu de la compréhension des actions et des passions des hommes [...] sa tâche est liée à la tâche que s'est donnée l'homme moderne de considérer l'existence comme le lieu problématique de la quête du sens'' (l.11 et l.16-17). Le roman permettrait donc de trouver un sens à la vie humaine, ce sens pouvant s'avérer difficile à saisir, ou apparaître de manière détournée. Cette quête du sens fait sans doute écho à un vieux précepte, le \textit{Gnothi seauton} (\textgreek{Γνῶθι σεαυτόν}) ou ``Connais-toi toi-même.''\footnote{Celon le \textit{Charmide} de Platon, il s'agit du plus ancien des trois préceptes qui furent gravés à l'entrée du temple de Delphes. Il assigne à l’homme le devoir de prendre conscience de sa propre mesure sans tenter de rivaliser avec les dieux\footcite{WikiGnothi}.}
			\subsubsection{Entre phénoménologie et existentialisme} \label{pheno}
				Avant de pouvoir comprendre sa propre existence nous dit D. Sallenave, il faut d'abord bien comprendre dans quel monde on vit, dans quel monde on fait l'expérience de la vie. C'est ce monde que le roman devra figurer. Dans notre extrait, l'auteure évoque ainsi un mystérieux ``monde de la vie'' (l.15) qui aurait une origine ``philosophique''. En fait, il s'agit sans doute d'une référence à la pensée Edmund Husserl, fondateur de la phénoménologie. Husserl, dans \textit{La crise des sciences européennes et la phénoménologie transcendantale} (aussi appelé \textit{Krisis} d'après le titre allemand), développe en effet la notion de \textit{Lebenswelt}, ou en français ``monde de la vie'', empruntée à Wilhelm Dilthey. Selon Nathalie Depraz dans son commentaire de Krisis\footcite{Depraz2012} :
				\begin{center}
					\footnotesize
					\begin{minipage}{0.7\textwidth}
						``Dès lors, le monde de la vie procède d’une attitude pré-théorique au sens de non idéalisante, et il se rapproche selon ce critère de l’attitude naturelle elle-même. [...] Le « monde de la vie » n’est donc pas le monde objectif de la science, il est bien plutôt le fondement oublié du sens de
						la science elle-même.''
					\end{minipage}
				\end{center}
				Le monde de la vie est donc le monde tel que nous le percevons, le ressentons, l'interprétons avec des moyens pré-scientifiques. C'est selon cette approche que le roman devrait être à même de donner du sens à la vie.\\
				Une fois défini le domaine auquel s'applique le roman, D. Sallenave nous explique sa portée étiologique en des termes qui héritent du vocabulaire existentialiste : ``sa tâche est liée à la tâche que s'est donnée l'homme moderne de considérer l'existence comme le lieu problématique de la quête du sens'' (l.16-17). Assez clairement, on retrouve ici la dialectique entre essence et existence développée par J.-P. Sartre notamment\footcite{Sartre1945}.
				\begin{center}
					\footnotesize
					\begin{minipage}{0.7\textwidth}
						``L'existentialisme athée, que je représente, est plus cohérent. Il déclare que si Dieu n'existe pas, il y a au moins un être chez qui
						l'existence précède l'essence, un être qui existe avant de pouvoir être défini par aucun concept et que cet être c'est l'homme ou, comme
						dit Heidegger, la réalité humaine. Qu'est-ce que signifie ici que l'existence précède l'essence? Cela signifie que l'homme existe
						d'abord, se rencontre, surgit dans le monde, et qu'il se définit après. L'homme, tel que le conçoit l'existentialiste, s'il n'est pas
						définissable, c'est qu'il n'est d'abord rien. Il ne sera qu'ensuite, et il sera tel qu'il se sera fait. Ainsi, il n'y a pas de nature humaine,
						puisqu'il n'y a pas de Dieu pour la concevoir. L'homme est non seulement tel qu'il se conçoit, mais tel qu'il se veut, et comme il se
						conçoit après l'existence, comme il se veut après cet élan vers l'existence, l'homme n'est rien d'autre que ce qu'il se fait. Tel est le
						premier principe de l'existentialisme.''
					\end{minipage}
				\end{center}
				L'existence est pour l'homme la donnée de départ. Mais l'existence ne présage en rien du sens de la vie d'un homme, autrement dit, elle ne dicte pas l'essence humaine. D'où une explication du caractère ``problématique'' de la quête du sens : l'homme, ``jeté'' dans la vie, y mène sa quête seul -- Sartre le dirait ``délaissé'' par les dieux -- avec pour seul appui le roman et son pouvoir de narration...
				ce qui rejoint la pensée de Lukacs\footcite{Lukacs1916}
				\begin{center}
					\footnotesize
					\begin{minipage}{0.7\textwidth}
						``L'épopée façonne une totalité de vie achevée par elle-même, le roman cherche à découvrir et à édifier la totalité secrète de la vie [...] Ainsi l'esprit fondamental du roman, celui qui en détermine la forme, s'objective comme psychologie des héros romanesques : ces héros sont toujours en quête [...]''\footcite[p.~54]{Lukacs1916}
					\end{minipage}
				\end{center}
				La ``totalité secrète de la vie'' est cette essence de l'homme qu'il s'agit de trouver.
			\subsubsection{Une impossible quête du sens ?}
				Cette quête du sens à travers le roman s'avère par conséquent essentielle mais paradoxale. Essentielle, elle l'est surtout depuis l'époque moderne; selon Dominique Rabaté\footcite{Rabate2010} en effet :
				\begin{center}
					\footnotesize
					\begin{minipage}{0.7\textwidth}
						``Cette capacité à figurer la vie d’un individu comme une particularité compréhensible
						et offerte à l’appréciation d’autrui me semble être ce que la littérature s’est justement proposée
						comme but et ambition à partir du XIXe siècle [...]''\footcite[p.~32-34]{Rabate2010}
					\end{minipage}
				\end{center}
				Pour Dominique Rabaté, la visée ontologique et explicative du roman constitue une ``mission'' récente. Mais cette mission est aussi considérée comme profondément ambiguë...
				\begin{center} 
					\footnotesize
					\begin{minipage}{0.7\textwidth}
						``Affirmer l’indéfini contre le défini, une vie contre la vie, c’est encore déjouer la quête d’un sens
						unique, contourner dès le départ l’écueil redoutable [...] de voir dans la littérature l’affirmation univoque d’une morale. C’est donc plutôt de l’impossibilité d’une réponse globale qu’il faudra partir ''
					\end{minipage}
				\end{center}
				Dominique Rabaté nous emmène ici sur le terrain de la morale, qui n'est pas exploré par D. Sallenave dans notre extrait. Cela dit, il met aussi l'accent sur la difficulté (voire la quasi-impossibilité) pour le roman de trouver un sens unique, une explication unique à l'existence.\\
				Le roman est constitutivement équivoque, un état de fait qui est repris dans notre texte : ``Si la littérature, si les Lettres, si le roman nous sont nécessaires, et même indispensables, c'est qu'en eux l'existence se donne comme un lieu du dévoilement toujours suspendu, toujours inachevé d'un sens. Les vérités du roman sont plurielles, et toujours recommencées [...]'' (l.17-20). Un autre exemple de cette pensée est à prendre chez Lukacs\footcite{Lukacs1916} :
				\begin{center}
					\footnotesize
					\begin{minipage}{0.7\textwidth}
						``Le roman est l'épopée d'un temps où la totalité extensive de la vie n'est plus donnée de manière immédiate, d'un temps pour lequel l'immanence du sens à la vie est devenue problème mais qui, néanmoins, n'a pas cessé de viser à la totalité''\footcite[p.~49]{Lukacs1916}
					\end{minipage}
				\end{center}
				On y retrouve cette idée du délaissement (cf. \ref{pheno}), et cette dissonance entre la visée (englobante, holistique) et le potentiel (restreint, ambigu) du roman. Est-ce là un défaut, une lacune du roman ? Pas vraiment nous dit D. Sallenave, qui considère cette incomplétude du roman comme source d'une richesse supplémentaire et d'un intérêt sans cesse renouvelé pour les récits des hommes.\\
				Ce paradoxe fécond propre au roman est par ailleurs soulevé par G. Lukacs\footcite{Lukacs1916} :
				\begin{center}
					\footnotesize
					\begin{minipage}{0.7\textwidth}
						``Le roman est l’épopée d’un monde sans dieux ; la psychologie
						du héros romanesque est démoniaque\footnote{ Notons que Lukacs évoque une dimension ``démonique'' du roman -- un concept inspiré de Goethe qui fait référence à une angoisse de l'homme face à un destin incertain, dicté par des lois non plus divines, mais chaotiques et imprévisibles. Le caractère disparate de la vie semble donc encore plus fort chez Lukacs que chez notre auteure.}, l’objectivité du roman, la virile et mûre
						constatation que jamais le sens ne saurait pénétrer de part en part la réalité et que
						pourtant, sans lui, celle-ci succomberait au néant et à l’inessentialité.'' \footcite[p.~94]{Lukacs1916}
					\end{minipage}
				\end{center}
				On retrouve dans ce passage des éléments déjà rencontrés : la relation entre le roman et l'épopée (cf. \ref{genealogie}), le caractère ``délaissé'' de l'homme en quête de sens (cf. \ref{pheno}), mais aussi bien sûr cette tension entre l'incomplétude essentielle du roman et la nécessité absolue de son existence pour ``faire tenir'' la réalité.
	\section{Roman et vie réelle : deux mondes distincts ?}
		Dans la première moitié de l'extrait que nous venons d'analyser, l'auteure nous présente le roman comme une image de la vie, qui s'avère salvatrice de par son éternelle perfectibilité. Dans la seconde moitié de notre extrait (l.21-34), c'est justement cette relation qu'entretient le roman avec la vie réelle qui est explicitée.\\
		D. Sallenave dépasse en effet la théorie du reflet pour une approche plus holistique, présentant le roman comme un produit ayant la vie comme plus pur matériau; mais aussi comme un remodelage possible de ce matériau.
		\subsection{Le roman comme partie de la vie}
			\subsubsection{Une essence partagée ?}\label{essencepartagee}
				Jusque là, D. Sallenave avait présenté le roman comme une image de la vie, qui, en tant qu'image, pouvait utiliser des supports, des matériaux différents de l'original. Toutefois, la nature réelle du roman est précisée au sein du troisième paragraphe (l.21-27). Et pour ce faire, l'auteure réutilise l'argument historique : ``si le récit est présent dès les origines de l'homme, c'est qu'il lui est en quelque sorte conaturel.'' (l.22-23). A l'adjectif ``conaturel'', on pourrait adjoindre l'adjectif ``consubstantiel'', ou, en termes logiques, ``dual''. En effet, la concomitance de l'apparition de la vie humaine d'une part et de celle du récit (le roman par extension) d'autre part prouve que l'une ne peut réellement exister sans l'autre, autrement dit, que la vie et le roman ont une essence commune. Selon M. Blanchot\footcite{Blanchot1959} par exemple :
				\begin{center}
					\footnotesize
					\begin{minipage}{0.7\textwidth}
						``Le récit n'est pas la relation de l'événement, mais cet événement même''
					\end{minipage}
				\end{center}
				La vie, l'événement et le récit de cet événement seraient intiment liés, intriqués : c'est du récit de la vie -- et non pas de la vie seule -- que naît l'événement tel qu'il nous apparaît. Comme dans une expérience quantique, toute opération de ``mesure'' de la vie (ici, le fait de la raconter), est à même de modifier ce fragment de vie lui-même. D. Sallenave semble également appuyer ce point de vue : ``En effet, innombrables sont les récits du monde\footnote{Notons que l'auteur auquel D. Sallenave fait explicitement référence est Roland Barthes\footcite{Barthes1966}, qui fut lui-même critique et théoricien :
					\begin{center}
						\footnotesize
						\begin{minipage}{0.7\textwidth}
							``Innombrables sont les récits du monde. C'est d'abord une variété prodigieuse de genres, eux-mêmes distribués entre des substances différentes, comme si toute matière était bonne à l'homme pour lui confier ses récits : le récit peut être supporté par le langage articulé, oral ou écrit, par l'image, fixe ou mobile, par le geste et par le mélange ordonné de toutes ses substances; il est présent dans le mythe, la légende, la fable, le conte, la nouvelle, l'épopée, l'histoire, la tragédie, le drame, la comédie, la pantomime, le tableau peint [...], le vitrail, le cinéma, les comics, le fait divers, la conversation.''
						\end{minipage}
					\end{center} L'``oubli'' du roman dans cet extrait est-il volontaire ? Barthes a-t-il voulu citer des supports ``exotiques'' pour en montrer la diversité, plutôt que des supports traditionnels ?}, et leur présence indique que chaque événement de la vie, pour avoir vraiment lieu, exige d'être raconté'' (l.25-26). La ``vraie vie'' est donc déplacée du domaine phénoménologique au domaine littéraire. Cette ``conversion'' de la vie vers le papier est expliquée par Pingaud\footcite{Pingaud1992} :
				\begin{center}
					\footnotesize
					\begin{minipage}{0.7\textwidth}
						``Dès l'instant où je raconte, ce que je raconte, arraché au temps vécu, ne fait plus qu'un avec le temps du récit [...] à partir du moment où on raconte, il n'y a plus, d'une certaine manière, de problème de contenu. C'est le contenu qui est lui-même devenu forme [...] comme tout ce que dit le narrateur, [les lacunes, les amplifications] font partie intégrante d'un temps qui n'est plus le temps abstrait du philosophe, ni non plus le temps concret de l'existence quotidienne, mais leur miraculeuse identification''
					\end{minipage}
				\end{center}
				Néanmoins, il faut noter que cette conversion ``miraculeuse'' du temps vécu en temps du récit ne l'est pas pour tout le monde. Bergson\footcite{Bergson1889} par exemple, grand penseur de la durée interne et du temps vécu, condamne (en général) la traduction de nos états de conscience en mots du langage :
				\begin{center}
					\footnotesize
					\begin{minipage}{0.7\textwidth}
						``Bref le mot aux contours bien arrêtés, le mot brutal qui emmagasine tout ce qu'il y a de stable, de commun et par conséquent d'impersonnel dans les impressions de l'humanité, écrase ou tout au moins recouvre les impressions délicates et fugitives de notre conscience individuelle [...] Ce qu'il faut dire, c'est que toute sensation se modifie en se répétant, et que, si elle ne me paraît pas changer du jour au lendemain, c'est parce que je l'aperçois maintenant à travers l'objet qui en est la cause, à travers le mot qui la traduit''
					\end{minipage}
				\end{center}
				Le fait de (se) raconter sa vie, selon Bergson, ne va pas sans un phénomène de compression, d'approximation, voire de falsification après un certain délai. Car pour lui, la vie relève du temps (la \textit{durée}, indifférentiable), et l'écriture, de l'espace (différentiable, énumérable, mesurable...).
			\subsubsection{Conséquences}
				Si le roman et la vie sont faits du même matériau, il s'ensuit que le roman est un moyen pour son auteur de pérenniser les moments de vie qui lui tiennent à cœur, mais que, en contrepartie, un ``bon roman'' doit revêtir, par mimétisme, une forme bien particulière.\\
				La première implication est sans doute la plus évidente; D. Sallenave l'évoque l.27 : ``Raconter, c'est d'abord faire être, et donc sauver de l'oubli et de l'inconsistance''.  Est rappelée en filigrane la dimension démiurgique de l'écrivain qui ``fait être'' un monde entier dans les pages de ses cahiers; mais surtout, est évoquée cette différence entre la vie et le roman qui est que l'une finit par vaciller tandis que l'autre perdure, se transmet. Cette volonté d'échapper par le récit au \textit{tempus fugit}\footnote{Cette expression a été relevée dans les \textit{Géorgiques} de Virgile (livre III, vers 284) : \textit{``Sed fugit interea, fugit irreparabile tempus, singula dum capti circumvectamur amore''}, ce qui signifie : ``Mais en attendant, il fuit : le temps fuit sans retour, tandis que nous errons, prisonniers de notre amour du détail''. »\footcite{WikiTempus}} nous renvoie, entre autres, au paradigme romantique.\\
				Toutefois, pour conserver telle quelle l'essence de la vie, le roman doit se plier, par mimétisme avec la structure même du temps, à un certain paradoxe formel : ``C'est ainsi que le récit s'offre comme le lieu d'une pensée de l'existence qui vise à son dévoilement'' (l.28-29), ou plus haut dans l'extrait : ``en eux [le roman, les Lettres...] l'existence se donne comme le lieu du dévoilement toujours suspendu'' (l.18-19). On note déjà le caractère réflexif du récit qui s'applique à se raconter lui-même, mais aussi cette tension inhérente entre le dire et le non-dire, entre écoulement et rétention de la matière temporelle. Un ``bon roman'', propre à rendre la profondeur de la vie, se doit d'être toujours à la limite entre ce qu'il véhicule en actes et ce qu'il garde en puissance. Voilà ce que pense aussi Bergson\footcite{Bergson1889}, qui nous l'avons vu (cf. \ref{essencepartagee}) dénonce la ``verbalisation'' de la vie en général, mais ne manque pas cela dit de donner sa vision idéale du roman :
				\begin{center}
					\footnotesize
					\begin{minipage}{0.7\textwidth}
						``Que si maintenant quelque romancier hardi, déchirant la toile habilement tissée de notre moi conventionnel, montre sous cette logique apparente une absurdité fondamentale, sous cette juxtaposition d’états simples une pénétration infinie de mille impressions diverses qui ont déjà cessé d’être au moment où on les nomme, nous le louons de nous avoir mieux connus que nous ne nous connaissions nous-mêmes.''
					\end{minipage}
				\end{center}
				Remarquons d'ailleurs que notre auteure et Bergson utilisent la même image du voile et du dévoilement pour figurer le processus de narration dans le roman. Ces idées ont été approfondies, dans le sillage de Bergson, et avec un langage très poétique, par Vladimir Jankélévitch\footcite{Jankelevitch1980}:
				\begin{center}
					\footnotesize
					\begin{minipage}{0.7\textwidth}
						``Le charme diffluent émane de la fluide continuité du devenir : c’est le charme de l’intervalle ; mais le charme de l’intervalle opère dans l’instant ; le charme, qu’il soit intervalle ou manière temporelle, s’appréhende comme une phosphorescence de l’instant et comme une apparition naissante-mourante.''\footcite[p.~113]{Jankelevitch1980}
					\end{minipage}
				\end{center}
				\begin{center}
					\footnotesize
					\begin{minipage}{0.7\textwidth}
						``L’occasion n’est pas l’instant d’un devenir solitaire, mais l’instant compliqué par le « polychronisme », c’est-à-dire par le sporadisme et le pluriel des durées.[...] La miraculeuse occasion tient à la polymétrie et à la polyrythmie, ainsi qu’à l’interférence momentanée des devenirs.''\footcite[p.~117]{Jankelevitch1980}
					\end{minipage}
				\end{center}
				Le roman, pour devenir réceptacle de la vie, doit se situer dans l'intervalle infinitésimal de la durée, et rendre l'interpénétration des états de conscience -- cette dernière opinion n'étant peut-être pas partagé par D. Sallenave, qui dit que narrer c'est au contraire ``mettre de l'ordre'' (l.28).
		\subsection{Le roman comme extension de la vie}
			\subsubsection{D'autres possibles pour le récit}
				Dans la dernière partir de notre extrait (l.29-34), D. Sallenave élargit les prérogatives du roman sur la vie. En effet, si le roman est susceptible de dire la vie, il est aussi susceptible \textit{a fortiori} de dire \textit{des vies} multiples et variées, comme autant ``mondes possibles''\footnote{la théorie des mondes possibles nous vient de Leibniz, et fit plus tard l'objet d'une formalisation par Kripke.} : ``Le roman est une pensée exploratoire des formes de l'existence parce qu'il en constitue des modèles expérimentaux, dont la succession infinie ouvre une interrogation jamais comblée [...] \textit{En figurant le monde}, la littérature l'ouvre au jeu, au rêve, à l'utopie, à l'uchronie'' (l.29-31). G. Lukacs\footcite{Lukacs1916}, dans l'extrait suivant, fait bien le lien entre la labilité formelle du roman et son potentiel explicatif :
			\begin{center}
				\footnotesize
				\begin{minipage}{0.7\textwidth}
					``Un tel péril ne peut être surmonté que si l'on pose comme ultime réalité, en pleine conscience et de façon parfaitement adéquate, ce que le monde a de fragile et d'inachevé, ce qui renvoie en lui à autre chose qui le dépasse''\footcite[p.~54]{Lukacs1916}
				\end{minipage}
			\end{center}
			Cette pluralité, ce dépassement dont parlent Lukacs et Sallenave, s'expriment à deux niveaux : au niveau génétique d'abord, avec l'existence de différents possibles pour un même récit; au niveau du contenu ensuite, avec la création de mondes parallèles au sein même du roman.\\
			La pluralité génétique n'apparaît qu'en filigrane de notre extrait. Pour D. Sallenave en effet, le roman a une dimension ``exploratoire'' et fait du monde ``un jeu''; cela est a mettre en perspective avec la pensée de la critique génétique et notamment de Jacques Petit, qui affirmait ``le texte n'existe pas'', ou de Michel Charles\footcite[p.~64]{Charles1985}, pour qui le texte pose problème d'une part à cause de ``[sa] fragilité, sous le double aspect de l'artifice de sa clôture (il est pris dans un réseau intertextuel) et de l'insécurité de a structure (l'énonciation le dynamise)'' et d'autre part à cause de ''l'insertion du texte dans une chaîne continue (genèse, réception)''. Tout roman est en effet en dialogue avec d'autres romans (c'est l'intertextualité) et avec un certain contexte d'écriture; il ne forme donc pas un monde clos sur lui-même. Mais surtout, il faut bien comprendre que ``le'' roman n'est en fait que la réalisation d'un possible parmi un halo d'autres possibles, (dont une part ne restera qu'au stade de brouillon, d'ébauche). En cela le roman est donc bien une expérience, un ``modèle'' possible de la vie.\\
			Mais c'est avant tout la pluralité de contenu qui est mise en avant par D. Sallenave. Plus précisément, cette pluralité s'affiche sur trois axes : l'axe de la fantaisie (``rêve''), l'axe géographique (``utopie''), l'axe temporel (``l'uchronie''). Cela fait écho à une phrase située plus haut dans l'extrait (``Même s'il rode aux confins de la science et de l'univers connu, même lorsqu'il s'empare d'objets fantastiques et qu'il s'emploie à décrire un monde autre [...]'' l.12-13). Selon Blanchot, le roman possède aussi ce pouvoir prospectif\footcite{Blanchot1959} :
			\begin{center}
				\footnotesize
				\begin{minipage}{0.7\textwidth}
					``ce pouvoir qui fait coïncider, en un même point fabuleux, le présent, le passé et même [...] l'avenir ''\footcite[p.~25]{Blanchot1959}
				\end{minipage}
			\end{center}
			Le roman est donc à même de dépasser le monde sensible en présentant, littéralement, ce qui n'existe pas dans la réalité, ce qui n'existe en aucun lieu et en aucun temps. Ainsi, pour reprendre l'analogie avec les  mondes possibles et la logique modale, le roman permet de passer de la vérité d'\textit{un} monde qui possède son modèle unique (sa ``vérité d'état''), à la vérité de mondes potentiels (dans passé, ou le futur), interconnectés pour certains, et possédant chacun leur propre modèle. Évidemment, cette conception du roman est relativement moderne, et on pense pour l'illustrer à l'œuvre de Jules Verne, d'Orwell ou de Lovecraft qui bien entendu s'avèrent plus ou moins proches de notre monde et délivrent un morale plus ou moins adéquate.
		\subsubsection{D'autres possibles pour la narration}
			Cette dernière section est une section d'ouverture venant compléter la théorie des mondes possibles du roman exposée jusqu'ici par D. Sallenave.\\
			En effet, il est fait état dans notre extrait des différents univers pouvant être déployés au sein d'un roman : univers parallèles, univers liés à des époques futures ou passées... Néanmoins, D. Sallenave n'explore pas ou peu les renouveaux possibles de la narration romanesque elle-même, en particulier la question de la voix, qui peut pourtant, elle aussi, initier le lecteur à des mondes nouveaux, et à une expérience inédite de la vie. A et égard, on peut imaginer deux mouvements inverses dans le changement de point de vue au sein du roman : la passage dans le monde infra-psychologique (avec Nathalie Sarraute), le passage dans le monde supra-psychologique (avec Virginia Woolf).\\
			Chez Sarraute, c'est le monde des tropismes qui est à l'œuvre dans la vie et les interactions des protagonistes. Le roman de Sarraute est un roman liminaire, à la limite du sensible et du conscient -- bien que l'auteure se soit refusé de donner une ``interprétation psychologisante'' à son œuvre.\\
			Un autre cas-limite est celui de Virginia Woolf, qui dans ses romans semble appréhender le monde en surplomb, comme une conscience flottante entre ses différents personnages, et qui réunit leurs pensées en une ``polyphonie muette''.\\
			Le monde construit par le roman, et l'image de la vie qu'il peut véhiculer, dépend donc aussi beaucoup de l'état d'esprit du romancier, comme en témoigne cet entretien avec Nathalie Sarraute\footcite{Causse2016} ou justement Virginia Woolf est évoquée :
			\begin{center}
				\footnotesize
				\begin{minipage}{0.7\textwidth}
					``Extraordinaire Woolf ! Je me souviens de cette phrase dans son \textit{Journal} : « Je creuse de belles grottes derrière mes personnages. »\\
					-- N’est-ce pas un avant-propos de vos \textit{Tropismes} ?\\
					-- Non. Pour moi, nos actes s’élaborent avant les mots. Je mets au jour la sensation, je la guide... Chez Virginia Woolf, l’espace et le temps se télescopent [...]\\
					-- Virginia Woolf écrit le ressenti des êtres, l’essence des choses, de la mer, des oiseaux, du ciel... ajouté-je.\\
					-- Mon amie Viviane Forrester affirme que chaque écrivain cherche la réalité qui le fait accéder à l’existence. Il y est tout entier engagé. Pour elle, chez Virginia Woolf le rythme des phrases capte le déroulé de l’instant. Son écriture s’est probablement élaborée au cours de l’enfance, au bord de la mer en Cornouailles, et à travers un processus inconscient lié aux événements tragiques qu’elle a vécus dans sa jeunesse...''
				\end{minipage}
			\end{center}
			On ressent dans ce passage les différences de point de vue entre les deux auteures, l'une préférant l'implicite, le roman ``sans les mots'', tandis que l'autre adopte une vision plus englobante de la réalité. La vie et le roman ont donc des définitions plurielles, qui dépendent aussi de la sensibilité de chacun.
			
			
		
	\section*{Conclusion}
		Dans cet extrait, Danièle Sallenave nous propose donc bien une poétique du roman, qui s'ancre dans la généalogie de cette forme littéraire. Héritant de l'objet de la tragédie et du mode de l'épopée, le roman est à même de se renouveler sans cesse, à l'image de la vie qu'il entend figurer et expliquer.\\
		Pour autant, cette mission que le roman s'est donnée n'est pas des plus simples, car il lui faut épouser la structure complexe et labile de la vie et par là même du temps vécu. De là cette dimension réflexive du roman, toujours ``en-train-de-se-faire'', à la limite du dit et du non-dit.\\
		Enfin, le roman pour accomplir sa mission est susceptible d'utiliser des moyens détournés, de nous emmener hors des frontières du réel pour mieux nous le faire saisir; cela est particulièrement vrai de la littérature moderne, qui pense l'avenir ou les possibles du présent comme jamais elle ne le fit auparavant auparavant. 
		

\medskip
\newpage
\bibliography{bibliography}
\footbibliography{footbibliography}
\bibliographystyle{frplain}
\footbibliographystyle{frplain}

\end{document}
