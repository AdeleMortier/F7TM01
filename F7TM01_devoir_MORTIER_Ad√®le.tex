\documentclass[a4paper,10pt]{article}
\usepackage[utf8]{inputenc}
\usepackage[french]{babel}
\usepackage{frbib}
\usepackage{french}
\usepackage{graphicx}
\usepackage{geometry}
\geometry{
	a4paper,
	left=20mm,
	top=20mm,
}
\usepackage{footbib}
\usepackage{textcomp}
\usepackage{graphicx}




\title{F7TM01 -- Théories et poétiques du récit\\
	 D. Sallenave, \textit{Le Don des morts. Sur la littérature}\\ \vspace{0.3cm}
	 \small Commentaire de texte}
\author{Adèle Mortier}

\begin{document}
	\maketitle
	\nocite{*}
	\section*{Introduction}
		Le roman est sans doute la forme de littérature qui nous est la plus intime, nous accompagnant en voyage, au coucher, ou durant nos trajets quotidiens. Si le roman nous parle plus que le théâtre ou la poésie, c'est peut-être et avant tout parce que le roman parle de nous; en tout cas c'est la thèse que soutient Danièle Sallenave dans son essai intitulé \textit{Le Don des morts. Sur la littérature}\footcite{Sallenave1991}, paru en 1991.\\
		Danièle Sallenave est tout à la fois une écrivaine de formation classique et membre de l'Académie française, une penseuse du communautarisme (religieux) et une chroniqueuse sur France Culture. Sa connaissance de la littérature antique l'a conduite à explorer les origines grecques du roman\footnote{alors même que le nom de ``roman'' pour désigner une composition littéraire est apparu autour du XIIème siècle, et tire son étymologie d’un adverbe latin, \textit{romanice}, ``à la manière des romains'', lui-même d’un adjectif classique \textit{romanus}...}, et à tisser des liens (formels, fonctionnels) entre roman, tragédie et épopée. Elle entend ainsi définir l'essence propre du roman, qui, selon elle ``n'est pas un genre littéraire", mais plutôt, comme nous allons le voir, un espace tout à la fois de réflexion de la vie, de réflexion sur la vie, et d'extension de la vie.\\
		Jouant sur la dialectique de l'existentialisme et du formalisme, de la description et de l'explication, de l'appartenance et de l'extension, Danièle Sallenave nous invite à nous poser les questions suivantes : en quoi une généalogie du roman permet-elle de le définir comme une forme transcendante ? Quelles sont les fonctions du roman, d'un point de vue philosophique ? Comment le roman se positionne-t-il par rapport à la vie ``réelle'' ? Notre plan suivra et détaillera se triple questionnement tout en soulignant certaines limites dans l'exposé de l'auteure.
		
		
	\section{Le roman comme ``forme'' littéraire}
		Dans la première partie de notre extrait, qui couvre la majorité du premier paragraphe (l. 1-11), D. Salenave adopte un point de vue généalogique et plutôt formaliste, le but étant de soutenir la thèse suivante : ``il y a une essence trans-historique du roman'' (l.3-4).
		\subsection{Situation du roman dans l'histoire littéraire : une généalogie\\ problématique}
			\subsubsection{Des origines discutables}
				Dès les premières lignes de l'extrait, D. Sallenave met en avant l'incertitude qui pèse sur les origines du roman : ``Qu'on veuille en dater l'émergence avec celle de l'homme moderne ou qu'on en fasse reculer l'apparition en des temps plus reculés avec le roman grec et latin, dont après tout la définition et les formes ne sont pas si éloignées des nôtres [...]'' (l.7-9).On note en particulier le grand écart historique qui sépare ces deux naissances hypothétiques : le roman dans son acception ``moderne'' (roman baroque ou roman picaresque) daterait du XVIème siècle ou du XVIIème siècle, tandis que le roman dans son acception ``antique'' remonterait environ au Ier siècle après J.-C. La première hypothèse, celle du roman ``moderne'' s'appuie davantage sur une parenté de contenu, tandis que la seconde, celle du roman ``antique'' ``dont après tout la définition et les formes ne sont pas si éloignées des nôtres [...]'' (l.9), aurait une justification de nature plus formelle.  En effet, le roman grec serait, dans le monde antique, la première forme de littérature en prose développant par ailleurs une intrigue cohérente et réaliste \footcite{WikiRomanGrec}.\\
				Néanmoins, D. Sallenave semble nous dire que cette incertitude temporelle importe bien peu, dans la mesure ou l'essence du roman est restée la même à travers les âges (en particulier, on pense à la veine baroque du roman moderne, fortement inspirée par le roman grec...). C'est cette conservation à travers les époques et malgré des changements de formes, des hybridations, qui semble donner au roman une dimension plus vaste que celle propre aux ``genres littéraires'' classiques comme la tragédie, évoquée l.4-5. Cette preuve de la portée ``trans-historique'' et par là même transcendante du roman prouve que ce concept dépasse le cadre des œuvres : c'est, dans l'approche formaliste, une ``forme littéraire'' plus qu'un genre littéraire.
			\subsubsection{Une généalogie paradoxale}\label{genealogie}
				Si le roman n'est pas un ``genre", l'auteure ne manque cependant pas d'évoquer les genres qui gravitent autour de lui, qui l'ont inspiré et/ou qui partagent avec lui certaines caractéristiques. Le roman est ainsi qualifié de ``successeur de l'épopée'' (l.10), et partage sa ``mission'' (l.10), une mission qui est aussi celle de la tragédie. Pour mieux situer de quoi il est question ici, on peut se référer à la lecture qu'ont fait Jean Lallot et Roselyne Dupont-Roc de \textit{La Poétique} d'Aristote\footcite{Lallot1980} :
				\begin{figure}[h]
					\begin{center}
						\includegraphics[width=150px]{aristote_genres.png}
						\caption{Parentés entre les trois principaux genres du théâtre antique. Note 1 du chapitre 5 de \textit{La Poétique} (édition du Seuil)}
					\end{center}
				\end{figure}
				
				Comme on peut le voir, tragédie et épopée sont en quelque sorte demi-sœurs, partageant même objet (un sujet élevé : dilemmes, passions des hommes nobles...), mais possédant des modes différents (mode dramatique pour la tragédie, narratif pour l'épopée). Lallot et Dupont-Roc ont également remarqué que la tragédie occupe dans ce diagramme une position polaire, partageant avec la comédie et l'épopée un caractère alors que comédie et épopée demeurent incommensurables. L'objet dont parlent Lallot et Dupont-Roc se rapproche de la ``mission'' évoquée par Sallenave, quoique le roman ne s'applique pas forcément à des sujets élevés. En outre, le roman partage avec l'épopée le mode narratif. Il serait donc un candidat possible pour rééquilibrer le diagramme et remplir une case vide de la typologie...\\
				Notons également qu'un telle approche mettant en regard mode t objet, forme et fonction, peut être mise en perspective avec celle du formalisme russe vers la fin des années 20. Pour Tynianov, ``l'étude des genres isolés est impossible hors du système dans lequel et avec lequel ils sont en corrélation"; le roman contemporain hériterait ainsi du roman grec de par sa forme prosaïque et de la poésie épique de par sa fonction. Or c'est à peu de choses près ce que nous dit D. Sallenave dans cet extrait.\\
				Cela dit, Sallenave reste sur un registre positiviste lorsqu'elle évoque la tragédie, qui dans notre généalogie s'avère être la tante du roman : ``Il n'en est pas du roman comme de la tragédie classique qui connut son temps d'éclat et finit par sombrer au milieu du XIIème siècle [...]'' (l.4-5). Or, dans l'optique de Chklovski et de sa théorie de la ``marche du cavalier"\footcite{Chklovski1973}, il se peut que la déliquescence de la tragédie ait précisément été causée par l'essor de la ``branche cadette", celle de l'épopée et du roman... ``L'art détruit son passé ou demeure vivant, tantôt s'embrasant, tantôt tombant dans la pitoyable existence d'une collection où tout a même valeur [...]"
		
		
	fonction qui change de forme
	
	Roman trns hitorique, transcendant, car c'est une forme rt pas un genre. Marche du cavalier avec l'épopée et la tragédie cf poly p38-39
	

	\section{Le roman et ses fonctions}
		Suite à cette étude du roman comme "essence", voyons comment cette essence se réalise et s'actualise pour remplir différents buts; bref, selon quelles modalités le roman \textit{existe}. Cette explication couvre la fin du premier paragraphe et le deuxième paragraphe (l.10-20).
		\subsection{Un vecteur de la \textit{mimésis}}
			\subsubsection{Un reflet de la vie}
				La mission la plus élémentaire du roman qui est détaillée dans notre extrait est de refléter la vie humaine : ``le roman ne cesse d'avoir pour tâche et pour sujet nous mêmes, notre existence dans le monde'' (l.13-14). ``Nous mêmes'' nous dit D. Sallenave, s'inspirant d'Aristote, c'est tout à la fois nos ``actions'' et nos ``passions'', ce que nous sommes en actes, et en puissance, notre intériorité et notre extériorité. Aristote expliquait en effet dans sa \textit{Poétique} : 
				\begin{center}
					\footnotesize
					\begin{minipage}{0.7\textwidth}
						``Une conséquence évidente de tout ceci, c'est que le poète doit être bien plus poète, ou compositeur, de fables que de vers, d'autant plus qu'il n'est réellement poète que parce qu'il imite des actions''\footcite[p.~51]{Aristote}
					\end{minipage}
				\end{center}
				Ici bien sûr Aristote ne parle pas du roman mais da la tragédie et de l'épopée. Cela dit, D. Sallenave comme on l'a vu considère le roman comme fils de l'épopée, celui-ci hérite donc directement de la fonction mimétique. 
				\begin{center}
					\footnotesize
					\begin{minipage}{0.7\textwidth}
						``L'épopée tient à la tragédie en ce qu'elle est comme elle, sauf le mètre, une imitation de actions nobles à l'aide du discours. Mais une différence, c'est que dans l'épopée le mètre est toujours le même, et qu'elle est toujours un récit.''\footcite[p.~27]{Aristote}
					\end{minipage}
				\end{center}
				Dans cet autre extrait est réaffirmée la parenté des ``missions'' entre l'épopée et la tragédie, malgré des différences de mode (cf \ref{genealogie}). D'un point de vue axiologique, Aristote juge la représentation positive, source de vertu pour l'homme.\\
				Plus proche de nous, Maurice Blanchot évoque aussi cet aspect mimétique du roman, à l'aide d'une métaphore nautique :
				\begin{center}
					\footnotesize
					\begin{minipage}{0.7\textwidth}
						``Avec le roman, ce qui est au premier plan, c'est la navigation préalable [...] cette navigation est une histoire toute humaine, elle intéresse le temps de hommes, est liée aux passions des hommes, elle a lieu réellement et elle est assez riche et assez variée pour absorber toutes les forces et toute l'attention des narrateurs''\footcite{Blanchot1959}
					\end{minipage}
				\end{center}
				On voit ici que l'accent est mis sur les passions humaines, mais aussi sur le ``réel''.\\
				Or, ce que D. Sallenave omet de mentionner dans notre extrait est que la représentation selon Aristote ne se doit pas d'être absolument fidèle au réel :
				\begin{center}
					\footnotesize
					\begin{minipage}{0.7\textwidth}
						``[...] le rôle du poète est de dire non pas ce qui a réellement eu lieu mais ce à quoi on peut s’attendre, [...]  la poésie dit plutôt le général, l’histoire le particulier'' (la poétique chap IX)''\footcite[chap.~IX]{Aristote}
					\end{minipage}
				\end{center}
				Il s'agit avant tout de créer une image de la vie qui soit certes cohérente, mais surtout harmonieuse. Cette notion de l'image ``harmonieuse'' mais non réelle, vivant par le \textit{muthos}, est néanmoins rejetée par Platon -- qui ne se fie qu'au \textit{logos} -- en particulier lorsqu'elle est présente dans l'épopée :
				\begin{center}
					\footnotesize
					\begin{minipage}{0.7\textwidth}
						``Or donc, poserons-nous en principe que tous les poètes, à commencer par Homère, sont de simples imitateurs
						des apparences de la vertu et des autres sujets qu’ils traitent, mais que, pour la vérité, ils n’y atteignent pas''\footcite{Platon}
					\end{minipage}
				\end{center}
				Cette double interprétation de la \textit{mimesis} est remarquablement expliquée par Myriam Revault d'Allonnes\cite{Revault1992} :
				\begin{center}
					\footnotesize
					\begin{minipage}{0.7\textwidth}
						``il [Aristote] contracte ou rétrécit la mimesis platonicienne en la faisant porter sur l'humain, et plus précisément encore, sur les hommes agissants et, faudrait-il ajouter, souffrants. En ce sens, on passe d'une \textit{mimesis} généralisée à une \textit{mimesis} restreinte, d'une \textit{mimesis} qui prend la mesure du réel en se référant (selon le critère de la ressemblance) à des formes stables et intelligibles, à une \textit{mimesis} qui se déploie dans le champ de l'agir et du pâtir humain. Chez Platon, la \textit{mimesis} est un mode de la participation, chez Aristote,
						elle explore le possible et donne à voir ce qui pourrait être autre qu'il n'est. ''
					\end{minipage}
				\end{center}
				Nous voyons donc que D. Sallenave ne soutient qu'un point de vue sur la \textit{mimesis}, qui fut largement discuté dès la période antique.
				%(+Hegel éventuellement? L'esthétique)
				%aristote : représentation des hommes ``agissants et souffrants"
				
				%Ce qu’Auerbach appelle style « créaturel » (kreatürlich), c’est le style qui se rapporte à l’être humain comme créature finie, sujette à la souffrance et à la mort. Ce style se développe chez Saint-François d’Assise et chez les ordres mendiants. Rabelais en y compris pour la truculence et la bouffonnerie. Il exprime la complexité contradictoire de la nature humaine, mais aussi le triomphe vitaliste et la pulsation dynamique de la réalité corporelle
				
			\subsubsection{\textit{Quid} du contexte ?}
				Dans cet extrait, D. Sallenave nous parle du roman et des hommes. Mais l'interprétation qui est faite de la \textit{mimesis} demeure très immanente : les hommes seraient caractérisés de façon satisfaisante par leurs principes internes (leur  passions) ou les manifestations directes des ces principes (leurs actions). La réalité à dépeindre n'est-elle pas plus complexe ? Les hommes ne sont-ils pas aussi influencés par des facteurs exogènes ?\\
				C'est en tout cas la thèse de l'un des grands théoriciens allemands du XXème siècle, Erich Auerbach. Auerbach\cite{Auerbach1946} prend notamment appui sur l'œuvre de Stendhal pour mettre en évidence les facteurs sociaux et historiques qui pèsent sur la théorie du reflet :
				\begin{center}
					\footnotesize
					\begin{minipage}{0.7\textwidth}
						``Dans la mesure où le réalisme sérieux des temps modernes ne peut représenter l’homme autrement qu’engagé dans une réalité globale
						politique, économique et sociale en constante évolution, Stendhal est son fondateur ''\footcite[p.~459]{Auerbach1946}
					\end{minipage}
				\end{center}
				Dans cette optique, les actions et les passions des hommes n'ont de sens que dans un contexte socio-historique défini, un contexte dans lequel l'homme est ``engagé''.
				\begin{center}
					\footnotesize
					\begin{minipage}{0.7\textwidth}
						``Le traitement sérieux de la réalité contemporaine, l’ascension de vastes groupes humains socialement inférieurs au statut de sujets d’une représentation problématique et
						existentielle, d’une part, - l’intégration des individus et des événements les plus communs
						dans le cours général de l’histoire contemporaine, l’instabilité de l’arrière-plan historique,
						d’autre part, - voilà, croyons-nous, les fondements du réalisme moderne'' \footcite[p.~487]{Auerbach1946}
					\end{minipage}
				\end{center}
				Dans cet autre passage, Auerbach semble adopter la thèse de Sallenave selon laquelle le roman véhicule un représentation ``existentielle'' des hommes, mais il nous dit aussi que cette thèse est incomplète : le roman -- ou à tout le moins le roman ``moderne'' -- doit aussi donner une idée du contexte dans lequel évoluent les personnages.
		
				Penser aussi au roman historique, qui est souvent caractérisé par la fusion des crises existentielles particulières et des crises politiques... Guerre et pais etc. Lukacs le roman historique, voire balzac aussi qui calque le social sur l'historique.
				Point de vue socai l Goldmann
		
		
			Sur la transposition et Aristote : Paolo Tortonese, l'homme en actions
		
		

			  
			
			
			
			
			
			
			

			
			
			
			
			
			
		\subsection{Un portée explicative contradictoire}
			citer Husserl, Sartre
			selon Rabaté\footcite{Rabate2010} :
			\begin{center}
				\footnotesize
				\begin{minipage}{0.7\textwidth}
					``Affirmer l’indéfini contre le défini, une vie contre la vie, c’est encore déjouer la quête d’un sens
					unique, contourner dès le départ l’écueil redoutable [...] de voir dans la littérature l’affirmation univoque d’une morale. C’est donc plutôt de l’impossibilité d’une réponse globale qu’il faudra partir ''
				\end{minipage}
			\end{center}
			Selon Rabate, aucune tentative d'explication ne pourra aboutir, ni ``explication", ni ``justification".
			
			selon Lukacs\footcite{Lukacs1916}
			 ``Le roman est l’épopée d’un monde sans dieux ; la psychologie
			 du héros romanesque est démoniaque, l’objectivité du roman, la virile et mûre
			 constatation que jamais le sens ne saurait pénétrer de part en part la réalité et que
			 pourtant, sans lui, celle-ci succomberait au néant et à l’inessentialité.'' p.94
			
			
			
			
			
			
			echo à l'ironie mentionnée par lukacs
			Parmi ces esthéticiens du début du romantisme, citons Goethe et la théorie du roman
			démonique. Le « démonique » évoque l’intranquillité, ce que le vingtième siècle appellera
			après Stendhal « l’ère du soupçon ». Les hommes de l’âge démonique perdent la sécurité du
			destin tracé par les dieux, aux dieux se substitue le démon, qui ressemble au hasard, à
			l’imprévisible, à l’aventure, à ce qui ne connaît ni causalité logique ni téléologie
	\section{Roman et vie réelle : deux mondes distincts ?}
		\subsection{Le roman comme partie de la vie}
			selon Pingaud\footcite{Pingaud1992}
			\begin{center}
				\footnotesize
				\begin{minipage}{0.7\textwidth}
					``Dès l'instant où je raconte, ce que je raconte, arraché au temps vécu, ne fait plus qu'un avec le temps du récit [...] à partir du moment où on raconte, il n'y a plus, d'une certaine manière, de problème de contenu. C'est le contenu qui est lui-même devenu forme [...] comme tout ce que dit le narrateur, [les lacunes, les amplifications] font partie intégrante d'un temps qui n'est plus le temps abstrait du philosophe, ni non plus le temps concret de l'existence quotidienne, mais leur miraculeuse identification''
				\end{minipage}
			\end{center}
			selon Blanchot\footcite{Blanchot1959}:
			\begin{center}
				\footnotesize
				\begin{minipage}{0.7\textwidth}
					``Le récit n'est pas la relation de l'événement, mais cet événement même''
				\end{minipage}
			\end{center}
			blanchot : intrication du temps vécu et du temps du récit (Proust) : le roman est monde de la vie.
			Blanchot voit le roman comme un temps à part, Sallenave comme un monde a part.
			
			
			
			selon Lukacs\footcite{Lukacs1916} p.54
			\begin{center}
				\footnotesize
				\begin{minipage}{0.7\textwidth}
					``Un tel péril ne peut être surmonté que si l'on pose comme ultime réalité, en pleine conscience et de façon parfaitement adéquate, ce que le monde a de fragile et d'inachevé, ce qui renvoie en lui à autre chose qui le dépasse'' p.54
				\end{minipage}
			\end{center}
		\subsection{Le roman comme extension de la vie}
		
		
		selon Rabaté\footcite{Rabate2010} : p.32-34
		\begin{center}
			\footnotesize
			\begin{minipage}{0.7\textwidth}
				``Cette capacité à figurer la vie d’un individu comme une particularité compréhensible
				et offerte à l’appréciation d’autrui me semble être ce que la littérature s’est justement proposée
				comme but et ambition à partir du XIXe siècle [...]''
			\end{minipage}
		\end{center}
		selon Lukacs\footcite{Lukacs1916} p.49
		\begin{center}
			\footnotesize
			\begin{minipage}{0.7\textwidth}
				``Le roman est l'épopée d'un temps où la totalité extensive de la vie n'est plus donnée de manière immédiate, d'un temps pour lequel l'immanence du sens à la vie est devenue problème mais qui, néanmoins, n'a pas cessé de viser à la totalité'' p.49
			\end{minipage}
		\end{center}
		
		selon Lukacs\footcite{Lukacs1916}
		\begin{center}
			\footnotesize
			\begin{minipage}{0.7\textwidth}
				``L'épopée façonne une totalité de vie achevée par elle-même, le roman cherche à découvrir et à édifier la totalité secrète de la vie [...] Ainsi l'esprit fondamental du roman, celui qui en détermine la forme, s'objective comme psychologie des héros romanesques : ces héros sont toujours en quête [...]'' p.54
			\end{minipage}
		\end{center}
		
		
		
		
			Pour un Formaliste, la littérature n’est pas une expression de la réalité, mais une réalité
			parallèle, artificielle, élaborée. 
			Brémond et le conte poly p43 
			
			Le récit se constitue de « séquences » qui sont
			formées d’éventualités, suivies ou non d’actualisations, il ouvre des pistes.
			selon Barthes\footcite{Barthes1966} :
			\begin{center}
				\footnotesize
				\begin{minipage}{0.7\textwidth}
					``Innombrables sont les récits du monde.C'est d'abord une variété prodigieuse de genres, eux-mêmes distribués entre des substances différentes, comme si toute matière était bonne à l'homme pour lui confier ses récits : le récit peut être supporté par le langage articulé, oral ou écrit, par l'image, fixe ou mobile, par le geste et par le mélange ordonné de toutes ses substances; il est présent dans le mythe, la légende, la fable, le conte, la nouvelle, l'épopée, l'histoire, la tragédie, le drame, la comédie, la pantomime, le tableau peint [...], le vitrail, le cinéma, les comics, le fait divers, la conversation.''
				\end{minipage}
			\end{center}
			On dépasse ce que Panofsky appelat  ‘imitations par copie'' [mimésis eikastiké], qui reproduisent les contenus de la réalité qui se donne à la perception sensible,
			ce qui revient à redoubler inutilement le monde matériel, qui n’est lui-même qu’une imitation des
			idées, selon Panofsky\footcite{Panofsky1983}
			
			
			selon Blanchot\footcite{Blanchot1959}
			\begin{center}
				\footnotesize
				\begin{minipage}{0.7\textwidth}
					``ce pouvoir qui fait coïncider, en un même point fabuleux, le présent, le passé et même [...] l'avenir '' p.25
				\end{minipage}
			\end{center}
			
			Ricoeur (inspiré de la phénoménologie), ipséité
			Jauss?
			Sarraute et les tropismes : niveau infrapsychologique!
			Woolf et le flux de conscience
			non évoqué : dialogue des oeuvres entre elles!
	\section{Discussion}
	\section*{Conclusion}








truc sur husserl (préface de depraz)
\begin{center}
	\footnotesize
	\begin{minipage}{0.7\textwidth}
		``Dès lors, le monde de la vie procède d’une attitude pré-théorique au sens de non idéalisante, et il se rapproche selon ce critère de l’attitude naturelle elle-même. [...] Le « monde de la vie » n’est donc pas le monde objectif de la science, il est bien plutôt le fondement oublié du sens de
		la science elle-même.''
	\end{minipage}
\end{center}

selon Blanchot\footcite{Blanchot1959}, accent mis sur la narration, le récit comme aventure humaine réelle. Aspect expérimental mis en avant également : le roman est une aventure, une Odyssée. Dimension de jeu absente de notre texte : aspect léger. Scepticisme fa ce au pouvoir de vérité du récit




Lukacs : d'accord sur le fait que le roman emprunte à l'épopée, mais que le contexte a changé.
Approche normative et exiologique de la tragédie et de l'épopée. Le roman est la continuation de l'épopée et hérite d'elle la grandeur que le vers lyrique n'est plus à meme de rendre

role du roman par rapport à l'épopée :

(Le monde de la convention n'est d'aucun secours, n'offre aucun réponse sur le sens de la vie t de actions. Or dans la littérature ne subsiste que ce qui est extra-conventionnel)
Exemple deDante qui opère la jonction entre epopee etroman, entre dépersonnalisation t individuation

Exemple de la biographie
\medskip
\newpage
\bibliography{bibliography}
\footbibliography{footbibliography}
\bibliographystyle{frplain}
\footbibliographystyle{frplain}

\end{document}
